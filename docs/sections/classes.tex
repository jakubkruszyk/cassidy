\section{Struktury}
\subsection{Struktury konfiguracyjne}
Wszystkie struktury konfiguracyjne oraz metody bezpośrednio z nimi związane znajdują się w pliku \emph{src/config.rs}
\newline\newline
\noindent Struktura \emph{Cli} zawiera pola odpowiadające wszystkim przełącznikom, których można użyć przy uruchamianiu programu. Sprawdzanie poprawności danych i konwersja na odpowiednie typy jest realizowana za pośrednictwem biblioteki \emph{Clap}.
\newline\newline
\noindent Struktura \emph{Config} zawiera wszystkie pola pliku konfiguracyjnego symulacji. Wczytywanie danych jest realizowane za pomocą bibliotek \emph{Serde} oraz \emph{toml}. Lista współczynników $\lambda$ jest reprezentowana wektorem struktur \emph{LambdaPoint}. Każda instancja zawiera dwa pola:
\begin{itemize}
\item time -- czas trwania danej fazy
\item coef -- współczynnik $\lambda$
\end{itemize}
\noindent Poprawność danych jest sprawdzana metodą \emph{validate()} zaimplementowaną dla struktury \emph{Config}.
\newline\newline
\noindent Struktura \emph{WalkOverConfig} zawiera wszystkie pola pliku konfiguracyjnego iteracji po zadanym parametrze. Wczytywanie i walidacja danych z pliku jest realizowana za pomocą bibliotek \emph{Serde} i \emph{toml}.


\subsection{Struktura użytkownika -- \emph{User}}
Struktura modelująca pojedynczego użytkownika oraz wszystkie zaimplementowane dla niej metody znajdują się w pliku \emph{src/user.rs}. Struktura zawiera unikatowy identyfikator (\emph{id}), czas pojawienia się użytkownika w systemie (\emph{start}) oraz czas zakończenia obsługi (\emph{end}), który jest sumą czasu pojawienia się i losowego czasu obsługi, losowanego z rozkładu równomiernego.
{
\fontencoding{T1}\selectfont 
\begin{minted}{rust}
pub struct User {
    pub id: usize,
    pub start: u64,
    pub end: u64,
}

impl User {
    pub fn new(id: usize, curr_time: u64, generator: &mut StdRng, cfg: &Config) -> User {
        // convert process_time from miliseconds to microseconds
        let delay: u64 =
            generator.gen_range((cfg.process_time_min * 1000)..=(cfg.process_time_max * 1000));
        User {
            id,
            start: curr_time,
            end: curr_time + delay,
        }
    }
}
\end{minted}
}

\subsection{Struktura stacji bazowej -- \emph{BaseStation}}
Struktura modelująca stację bazową i zaimplementowane na niej metody znajdują się w pliku \emph{src/basestation.rs}. Każda instancja stacji zawiera unikatowy identyfikator (\emph{id}), kopiec binarny reprezentujący bloki zasobów (\emph{resources}), znacznik czasu pojawienia się kolejnego użytkownika (\emph{next\_user\_add}), stan stacji (\emph{state}) oraz liczniki zużytej mocy (\emph{total\_power}), zużytych zasobów (\emph{total\_usage}) i czasu w trybie uśpienia (\emph{sleep\_time}).

{
\fontencoding{T1}\selectfont 
\begin{minted}{rust}
pub struct BaseStation {
    pub id: usize,
    resources: BinaryHeap<User, FnComparator<fn(&User, &User) -> Ordering>>,
    pub next_user_add: u64,
    pub state: BaseStationState,
    pub total_power: f64,
    pub total_usage: f64,
    pub sleep_time: u64,
}
\end{minted}
}

\noindent Stan stacji bazowej jest reprezentowany jednym z wariantów typu wyliczeniowego (ang. \emph{enum}) \emph{BaseStationState}. Stany \emph{PowerUp} i \emph{PowerDown}, reprezentujące stan przełączenia pomiędzy stanami aktywnym i uśpienia, przechowują znacznik czasu zakończenia tego stanu. Zdarzenia generowane przez stację są reprezentowane przez typ wyliczeniowy \emph{BaseStationEvent}.
{
\fontencoding{T1}\selectfont 
\begin{minted}{rust}
pub enum BaseStationState {
    Active,
    Sleep,
    PowerUp(u64),   // Station is during power-up process
    PowerDown(u64), // Station is during power-down process
}

pub enum BaseStationEvent {
    ReleaseUser,
    AddUser,
    PowerUp,
    ShutDown,
}
\end{minted}
}

\noindent Po zakończeniu symulacji, zawartości liczników są dzielone przez czas trwania symulacji, a następnie są zwracane jako instancja struktury \emph{BaseStationResult}.
{
\fontencoding{T1}\selectfont 
\begin{minted}{rust}
pub struct BaseStationResult {
    pub average_power: f64,
    pub average_usage: f64,
    pub average_sleep_time: f64,
}
\end{minted}
}

\subsection{Struktura kontenera symulacji -- \emph{SimContainer}}
Struktura \emph{SimContainer}, zdefiniowana w pliku \emph{src/sim\_container.rs}, jest abstrakcyjną strukturą modelującą system. Zawiera jedynie instancje struktur konfiguracyjnych. Stan symulacji, modelowany strukturą \emph{SimState}, jest tworzony osobno dla każdego przebiegu symulacji. Niezależność przebiegów umożliwia proste zrównoleglenie obliczeń za pomocą biblioteki \emph{Rayon}. Dwie najważniejsze metody zaimplementowane na strukturze \emph{SimContainer} to:
\begin{itemize}
\item \emph{simulate()} -- tworzy nową instancję stanu i uruchamia jeden przebieg symulacji. Wyniki są zwracane jako instancja struktury \emph{SimResults}, które jest zdefiniowana w pliku \emph{src/sim\_container/sim\_results.rs},
\item \emph{run()} -- wywołuje równolegle metodę \emph{simulate()} a następnie uśrednia uzyskane wyniki, które są zwracane jako pojedyncza instancja struktury \emph{SimResults}, która zawiera uśrednione wyniki z całego systemu oraz wektor wyników z poszczególnych stacji
\end{itemize}
{
\fontencoding{T1}\selectfont 
\begin{minted}{rust}
pub struct SimContainer {
    cli: Cli,
    cfg: Config,
}

pub struct SimState {
    pub time: u64,
    pub next_user_id: usize,
    pub lambda: f64,
    pub lambda_update_time: u64,
    pub lambda_update_idx: usize,
    pub all_users: usize,
    pub redirected_users: usize,
    pub dropped_users: usize,
}

pub struct SimResults {
    pub average_usage: f64,
    pub average_power: f64,
    pub average_drop_rate: f64,
    pub total_users: usize,
    pub dropped_users: usize,
    pub stations: Vec<BaseStationResult>,
}
\end{minted}
}
