\section{Treść zadania}
Rozważmy system radiokomunikacyjny składający się z N stacji bazowych posiadających R bloków zasobów (ang. Resource Blocks). W losowych odstępach czasu $\tau$ (wynikającej z intensywności zgłoszeń $\lambda$) w każdej stacji bazowej pojawiają się użytkownicy. Każdy użytkownik zajmuje jeden bloków na losowy czas $\mu$. Jeśli stacja bazowa nie ma wystarczającej liczby bloków zasobów by obsłużyć użytkownika jego zgłoszenie może być przekierowane do sąsiedniej stacji. Jeśli żadna ze stacji bazowych nie może obsłużyć zgłoszenia jest ono tracone. Intensywność zgłoszeń w systemie zmienia się cyklicznie: przez pierwsze 8 godziny intensywność zgłoszeń wynosi $\lambda / 2$ przez kolejne 6 godzin - $3\lambda / 4$, następnie przez 4 godziny wynosi $\lambda$, po czym spada do wartości $3\lambda / 4$ na 6 godzin i cykl się powtarza. Dla stacji bazowych można ustalić próg przejścia w stan uśpienia L (wyrażony w \% zajętych bloków zasobów). Stacja bazowa w stanie uśpienia pobiera moc równą 1 W, a podczas gdy jest aktywna 200 W. Zgłoszenia z uśpionej stacji są przejmowane równomiernie przez sąsiednie stacje. Podobnie jeśli w jednej z sąsiednich komórkach przekroczony zostanie próg H (wyrażony w \% zajętych bloków zasobów), uśpiona komórka jest aktywowana i przejmuje połowę zgłoszeń ze stacji, w której przekroczony został próg H. Proces uśpienia i aktywacji komórki trwa 50 ms i zużywa jednorazowo 1000 W.