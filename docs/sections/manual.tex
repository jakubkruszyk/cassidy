\section{Instrukcja użytkowania}
\subsection{Uruchamianie programu}
Program \emph{cassidy} jest uruchamiany z wiersza poleceń w następujący sposób:
{
\fontencoding{T1}\selectfont 
\begin{minted}{shell}
cassidy [Options] --duration <f64>
\end{minted}
}

\noindent Dostępne opcje przedstawiono w tabeli \ref{options_table}.
\newline\newline
\begin{table}[h]
\caption{Dostępne przełączniki programu \emph{cassidy}}
\label{options_table}
\begin{center}
\fontencoding{T1}\selectfont
\renewcommand{\arraystretch}{1.5}
\begin{tabular}{|l|l|}
\hline 
Opcja & Opis \\ 
\hline 
-{}-with-config & Ścieżka do pliku konfiguracyjnego symulacji \\ 
\hline
-{}-seed <u64> & Ziarno generatora liczb losowych. [domyślnie: entropia systemu] \\
\hline
-{}-log & Aktywuje generowanie pliku event log \\
\hline
-{}-duration <time> & Czas trwania symulacji w godzinach \\
\hline
-{}-iterations <u32> & Liczba iteracji uśredniających wynik końcowy. [domyślnie: 1]\\
\hline
-{}-enable-sleep & Aktywuje logikę odpowiadającą za usypianie i wybudzanie stacji bazowych \\
\hline
-{}-save-default-config <path> & Domyślna konfiguracja symulacji zostanie zapisana pod ścieżką \emph{path} \\
\hline
-{}-show-partial-results & Aktywuje wyświetlanie wyników każdej iteracji \\
\hline
-{}-log-wave & Aktywuje generowanie pliku log w formacie binarnym \\
\hline
-{}-samples <u32> & Dzielnik ilości zapisywanych próbek do binarnego pliku log [domyślnie: 1] \\
\hline
-{}-walk-over <path> & Aktywuje iterowanie po zadanym parametrze z pliku pod ścieżką \emph{path} \\
\hline
-h, -{}-help & Wyświetla pomoc \\
\hline
-V, -{}-version & Wyświetla wersję programu  \\
\hline
\end{tabular}
\end{center}
\end{table}

\noindent \textbf{Przykłady uruchamiania}:
\newline\newline
Pojedyncza iteracja, z domyślną konfiguracją przez 24 godziny czasu symulacji

{
\fontencoding{T1}\selectfont 
\begin{minted}{shell}
cassidy --duration 24
\end{minted}
}

\noindent Dziesięć iteracji, z konfiguracją wczytaną z pliku \emph{my\_cfg.toml} przez 24 godziny czasu symulacji

{
\fontencoding{T1}\selectfont 
\begin{minted}{shell}
cassidy --duration 24 --iterations 10 --with-config my_cfg.toml
\end{minted}
}

\noindent Dziesięć iteracji, z konfiguracją wczytaną z pliku \emph{my\_cfg.toml} przez 24 godziny czasu symulacji, z włączonym trybem uśpienia oraz wyświetlaniem wyników cząstkowych

{
\fontencoding{T1}\selectfont 
\begin{minted}{shell}
cassidy --duration 24 --iterations 10 --with-config my_cfg.toml --enable-sleep \
	--show-partial-results
\end{minted}
}

\noindent Pojedyncza iteracja, z konfiguracją wczytaną z pliku \emph{my\_cfg.toml} przez 24 godziny czasu symulacji, dla każdej wartości parametru zdefiniowanego w pliku \emph{my\_walk\_cfg.toml}

{
\fontencoding{T1}\selectfont 
\begin{minted}{shell}
cassidy --duration 24 --with-config my_cfg.toml --walk-over my_walk_cfg.toml
\end{minted}
}

\subsection{Pliki log}
Program \emph{cassidy} może wygenerować jeden lub więcej z czterech typów pliku log:
\begin{enumerate}
\item gdy przełącznik \emph{-{}-walk-over} NIE jest użyty, wyniki symulacji, w formacie czytelnym dla człowieka, są wyświetlane oraz zapisywane do pliku \emph{sim\_report}
\item gdy przełącznik \emph{-{}-walk-over} jest użyty, wyniki symulacji, w formacie \emph{csv}, są zapisywane do pliku \emph{sim\_report}. Pierwsza linia zawiera nazwy zapisywanych sygnałów, a pierwsza kolumna zawsze zawiera wartość parametru wymienionego w pliku konfiguracyjnym
\item gdy przełącznik \emph{-{}-log} jest użyty, każdy \emph{event} wykonany w symulacji jest zapisywany do osobnego pliku log pod ścieżką "sim.run\_[run\_no]\_no\_[iteration\_no]"
\item gdy przełącznik \emph{-{}-log-wave} jest użyty, po każdym wykonanym \emph{event'cie} zużycie zasobów oraz stan każdej stacji są zapisywane do osobnego pliku log pod ścieżką "sim\_bin.run\_[run\_no]\_no\_[iteration\_no]"
\end{enumerate}

\subsection{Pliki konfiguracyjne}
\subsubsection{Konfiguracja symulacji}
Plik konfiguracji symulacji, w formacie \emph{toml}, zawiera wszystkie parametry symulacji oprócz jej całkowitego trwania. Poniżej znajduje się domyślna konfiguracja, którą można również uzyskać za pomocą przełącznika \emph{-{}-save-default-config}:

{
\fontencoding{T1}\selectfont 
\begin{minted}{toml}
process_time_max = 15000 # Górny zakres czasu przetwarzania użytkownika
process_time_min = 1000  # Dolny zakres czasu przetwarzania użytkownika
lambda = 10.0            # Średnia liczba napływających użytkowników na sekundę w każdej
                         #   stacji bazowej. Ta wartość będzie przemnożona przez
                         #   współczynnik z listy lambda_coefs
resources_count = 273    # Liczba bloków zasobów w każdej stacji
sleep_threshold = 20     # Próg z zakresu <0, 100>%. Jeżeli zużycie zasobów stacji będzie
                         #   poniżej tego progu, stacja spróbuje przejść w stan uśpienia
wakeup_threshold = 80    # Próg z zakresu <0, 100>%. Jeżeli zużycie zasbów stacji będzie
                         #   powyżej tego progu, system spróbuje wybudzić jedną uśpioną stację
stations_count = 10      # Liczba stacji bazowych w systemie
active_power = 200.0     # Moc pobierana przez stację w stanie aktywnym
sleep_power = 1.0        # Moc pobierana przez stację w stanie uśpienia
wakeup_power = 1000.0    # Jednostkowa moc pobierana przez stację, która jest przełączana
                         #   ze stanu uśpienia do aktywnego i vice versa
wakeup_delay = 50        # Opóźnienie przełączania stacji ze stanu uśpienia do aktywnego
                         #   i vice versa
log_buffer = 10000       # Rozmiar bufora loggera. Obecnie to ustawienie nie ma znaczenia

[[lambda_coefs]]         # Lista par (współczynnik lambda, czas trwania)
time = 8.0               # Czas trwania, wyrażony w godzinach, tej fazy
coef = 0.5               # Współczynnik lambda tej fazy

[[lambda_coefs]]
time = 6.0               # Czas trwania fazy jest względny, dlatego ta faza będzie trwać
                         # przez 6 godzin po zakończeniu poprzedniej fazy. W tym przykładzie
                         # będzie to zakres od 8 godziny do 14 godziny czasu symulacji
coef = 0.75

[[lambda_coefs]]
time = 4.0               # Jeżeli całkowity czas trwania symulacji jest dłuższy od sumy
                         # czasów trwania wszystkich faz, cykl rozpocznie się od początku
coef = 1.0

[[lambda_coefs]]
time = 6.0
coef = 0.75
\end{minted}
}

\subsubsection{Konfiguracja iteracji}
Plik konfiguracji iteracji, w formacie \emph{toml}, zawiera nazwę parametru, po którym program będzie iterował, oraz zakres jego wartości. Program nie posiada domyślnej konfiguracji i użytkownik musi zapewnić własną.

{
\fontencoding{T1}\selectfont 
\begin{minted}{toml}
var = "Lambda"  # Nazwa zmiennej po której nastąpi iteracja. Możliwe wartości to Lambda,
                #   SleepLow (sleep_threshold) oraz SleepHigh (wakeup_threshold)
start = 10.0    # Początkowa wartość zmiennej
end = 50.0      # Końcowa wartość zmiennej
step = 5.0      # Wartość która zostanie dodana do parametru po każdej iteracji.
                # W tym przykładzie symulacja zostanie wykonana dla wartości lambda:
                # [10, 15, 20, 25, 30, 35, 40, 45, 50]
\end{minted}
}

\subsection{Skrypty pomocnicze}\label{python_scripts}
Aby ułatwić pracę z programem, przygotowano skrypty w języku \emph{Python} do wizualizacji danych zapisanych w \emph{logach} generowanych przez program. Skrypty znajdują się w folderze \emph{scripts/} w załączonym repozytorium.

\subsubsection{parse\_bin\_log.py}
Skrypt służący do wyświetlania binarnych plików log generowanych przełącznikiem \emph{-{}-log-wave}.
{
\fontencoding{T1}\selectfont 
\begin{minted}{shell}
Usage: python parse_bin_log.py <path_to_log> [subsampling]
\end{minted}
}

\noindent Przykład wczytania co setnej próbki z pliku \emph{sim.log}.
{
\fontencoding{T1}\selectfont 
\begin{minted}{shell}
python parse_bin_log.py sim.log 100
\end{minted}
}
\noindent Domyślnie zostaną wyświetlone dane tylko z pierwszej stacji. Aby wyświetlić/ ukryć dane z pozostały stacji należy klikną na odpowiednią linię na legendzie wykresu.

\subsubsection{parse\_csv\_log.py}
Skrypt służący do wyświetlania zawartości z raportu symulacji w formacie \emph{csv} generowanego przełącznikiem \emph{-{}-walk-over}.
{
\fontencoding{T1}\selectfont 
\begin{minted}{shell}
Usage: python parse_csv_log.py <path_to_log>]
\end{minted}
}

\subsubsection{rng\_histograms.py}
Skrypt do wyświetlania wyników generatorów liczb losowych czasu przetwarzania użytkownika oraz czasu do pojawienia się następnego użytkownika. Aby wygenerować dane z generatorów należy użyć poniższej komendy:
{
\fontencoding{T1}\selectfont 
\begin{minted}{shell}
cargo test -- test_rng generate_lambda
\end{minted}
}
\noindent Następnie aby wyświetlić wykresy należy użyć poniższej komendy z folderu \emph{scripts/}:
{
\fontencoding{T1}\selectfont 
\begin{minted}{shell}
python rng_histograms.py
\end{minted}
}